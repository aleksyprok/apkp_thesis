This thesis used mathematical models to develop our understanding of MHD waves in the solar corona.
We chose our models to be complex enough to allow the phenomena we are interested in to be studied and simple enough to be easily understood. We calculated analytic solutions and then verified many of these results numerically by comparing them side-by-side using graphs. The work presented here may play a role in solving the coronal heating problem and developing coronal seismology. 

Chapter \ref{chap:ideal_footpoint_driven_alfven_waves} introduced some of the key concepts relevant to the rest of this thesis. We modelled linear ideal Alfv\'en waves which we injected into the corona by using a footpoint driver. The waves were modelled as perturbations on a uniform background magnetic field and Alfv\'en speed. To begin with, we used a sinusoidal driver in a closed loop and found that the form of the solution was highly dependent on the driver frequency. If the driver frequency equalled one of the natural Alfv\'en frequencies of the loop, then resonance occurred. This meant that the energy grew quadratically with time. If the driver frequency was not equal to a natural frequency, then the energy oscillated about a finite value at the beating frequency (which goes to zero as the driver frequency tends towards natural frequencies). We showed that a white and red noise force driver causes the energy to grow linearly with time on average. A white noise driver is a random driver which excites all frequencies with equal force, and a red noise driver has a power spectrum which goes as $f^{-2}$, where $f$ denotes frequency. In Section \ref{sec:noisy_force} we give a more precise definitions. Observational evidence (see Figure \ref{fig:power_spectrum_morton}) suggests the power spectrum slope for waves in the corona lies somewhere in between $f^{-2}$ and $f^0$. We also considered the case where the waves were partially confined in a leaky loop where a fraction $R<1$ of the incident wave amplitude reflects at the boundary. We used a sinusoidal driver and found that leakage prevents energy of the system from growing to infinity, and the system converges towards a steady-state. At steady-state, the amplitude of the waves is fixed, and the system oscillates at the driver frequency. Note that a system will also converge towards a steady-state if resistivity or viscosity is included in the model.

In Chapter \ref{chap:resistive_phase_mixed_alfven_waves}, our goal was to answer whether the dissipation of phase-mixed Alfv\'en waves plays an essential or negligible role in coronal heating. To achieve this goal, we introduced a term called the damping rate per unit of wave energy which we denote $\gamma$. We estimated that for phase mixing to be a viable heating mechanism then our phase mixing model needs to have a $\gamma$ of about $10^{-1}\si{.s^{-1}}$. We are careful to ensure any simplifications we make act to increase $\gamma$, which means our calculation for $\gamma$ is an upper bound. For example, we assume the system is at steady-state which acts to increase $\gamma$, in \citet{Arregui2015} they argue that Alfv\'en waves may not have time to reach steady-state due to thermodynamic changes in the loop. We first introduced phase mixing in an open-loop and calculated the analytic solution using the method of multiple scales. Neighbouring field lines had different natural frequencies, which meant steep gradients formed perpendicular to the velocity and magnetic fields. After that, we modelled a closed-loop and extended our solution for the open-loop using a method of images approach to calculate the analytic solution. We included leakage into the model and found that this acts to reduce $\gamma$. The results from this chapter suggest that the dissipation of the gradients produced by phase mixing plays a negligible role in coronal heating. However, it is possible that phase mixing plays a less direct role; for example, it could trigger the Kelvin-Helmholtz and tearing mode instabilities which cause a turbulent cascade. In this case, the dissipation will occur primarily due to gradients parallel to the velocity and magnetic fields because the $\vec{W}^{(0)}$ component of the viscosity tensor will dominate. Note that the gradients formed by phase mixing are perpendicular to the velocity field and magnetic field.

In the resonant absorption literature, authors typically assume the background magnetic field is perpendicular to the transition region. \citet{Halberstadt1993,Halberstadt1995,Arregui2003} use line-tied boundary conditions in their resonant absorption models to show that if the background magnetic field is oblique to the transition region then steep boundary layers/evanescent fast waves can form. The goal in Chapter \ref{chap:resonant_absorption_in_an_oblique_field} was to check that these boundary layers are physical and not a result of approximating the steep jump in density from the corona to the chromosphere with line-tied boundary conditions. To test this, we first introduced resonant absorption in a model where the background field is normal to the transition region and the Alfv\'en speed was purely a function of $x$ to show some of the key properties of resonant absorption. We showed resonant absorption occurs where the frequency of the incoming fast waves equals the natural Alfv\'en frequency of a resonant field line resulting in the formation of singularities at discrete $x$-coordinates. At the singularities, the mode conversion from fast waves to Alfv\'en waves is one-way and is caused by gradients in the magnetic pressure. After that, we modified the model to let the background Alfv\'en speed be uniform in $x$ and piecewise constant in $z$, where $z>0$ models the corona and $z<0$ corresponds to the chromosphere. This meant resonant absorption could not occur. However, we derived dispersion relations and simulated the singularities which would form in a resonant absorption experiment by letting $k_x\rightarrow \infty$. We used an asymptotic expansion to express the solutions in a simpler form for $k_x\rightarrow \infty$. If line-tied boundary conditions are used then $u_x$ to leading order contains a steep evanescent fast wave/boundary layer term to leading order. However, if the chromosphere is included in the model then the steep boundary layers do not appear in the leading order expansion for $u_x$. This suggests that imposing line-tied boundary conditions can cause the model to significantly overestimate the amplitude of the boundary layers. Finally, we extended the model to let the Alfv\'en speed have an $x$-dependence. By using an eigenfunction approach we showed that near singularity/resonant location the boundary layers are not present to leading order. This helps to confirm that line-tied boundary conditions can cause significant overestimation of the size of the boundary layers. However, line-tied boundary conditions can still be useful in future models for their simplicity, provided their limitations are understood. In the future, we would like to compare resonant absorption experiments where line-tied boundary conditions and where the chromosphere is included in the model to precisely quantify the effects of imposing line-tied boundary conditions in a resonant absorption experiment. Also, we would like to extend our results from the Cartesian coordinate system to a cylindrical coordinate system as this can more easily be applied to coronal loops.

This thesis focused on improving our understanding of MHD waves and pushing the boundaries of our current knowledge. We hope that results from the work presented here lead to developments in coronal seismology. In the future, we aim to apply contemporary MHD wave theory to observational data to infer approximate values for difficult to measure quantities, e.g. the magnetic field strength and density (see \citealt{Mathioudakis2013}). Calculating the values of quantities in the corona could lead to significant developments in explaining the observed dynamics and may present us with new puzzles to solve. With ever-improving computational power and observational instruments, coronal seismology is becoming an increasingly important and relevant area of research. For coronal seismology to become more widely used, we need to develop more user-friendly code and software to automate much of the process. We hope to create software and code that enables users to estimate quantities using coronal seismology quickly. Ideally, the code and software would be easy to use such that the user does not need extensive knowledge of MHD wave theory to estimate the desired quantities. 

Another suggestion for future work is to investigate if the viscous dissipation of fast waves plays a significant role in the quiet sun. In \citet{Withbroe1977,Parker1991} they suggest that the waves need to dissipate over a length scale of 1 or 2 solar radii to heat the corona. In Appendix \ref{adx:coronal_heating_by_viscous_fast_waves}, we use a simple toy model to show that the fast waves can dissipate over short enough length-scales. However, we make many simplifications and so a more extensive analysis is required. 

It has been a privilege to have had this opportunity to contribute towards solar physics research. We hope that the readers found this thesis interesting and useful for developing their understanding of solar/plasma physics. Acquiring a better knowledge of solar/plasma physics at a theoretical level may play a key role in solving problems such as the coronal heating problem which could reveal new and exciting problems which need to be solved. Developments in coronal heating could lead to results in other areas such as nuclear fusion due to our improved understanding of plasma physics.  Finally, it could be relevant for space weather prediction,  which is increasingly important with our ever-increasing dependence on orbital satellites.