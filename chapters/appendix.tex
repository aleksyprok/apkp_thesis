In this appendix, we expand upon a proposed topic for further research mentioned in Chapter \ref{chap:conclusions_and_future_work}. Namely, the possibility that the viscous dissipation of fast waves plays a major role in coronal heating. We will present a simple toy model to show our proposal's plausibility and then discuss how to investigate this rigorously in the future.

\section{Model and assumptions}

In this section, we will describe the model we will use. Our domain is designed to approximate the quiet sun conditions near the equator. We model perturbations on a static background equilibrium and make the same assumptions as those described by Equation \eqref{eq:linear_assumotion_v}-\eqref{eq:linear_assumotion_rho}.The plasma is modelled as cold, i.e. $\beta=0$, which means the magnetic pressure force is negligible. We assume the Lorentz force dominates and neglect the gravitational force. Our background quantities are assumed to be uniform with the background magnetic field pointing south to north and approximately tangential to the solar surface at the equator,
\begin{equation}
    \vec{B}_0 = B_0\vec{\hat{z}},
\end{equation}
The $\vec{\hat{x}}$ direction points radially outwards from the solar surface.

We model the linearised momentum equation as
\begin{equation}
    \pdv{\vec{u}}{t}=v_A^2\qty[\pdv{\vec{\hat{b}}}{z}-\grad \hat{b}_z] + \frac{1}{\rho}\div{\vec{\pi}_0},
\end{equation}
and the linearised induction equations as
\begin{equation}
    \pdv{\vec{\hat{b}}}{t}=\pdv{\vec{u}}{z} - \div{\vec{u}}\,\vec{\hat{z}},
\end{equation}
where $\vec{\pi}_0$ denotes the linearised viscosity tensor (see Equation \ref{eq:braginskii_viscous_stress_tensor}) and
\begin{equation}
    \vec{\hat{b}} = \frac{\vec{b}}{B_0}.
\end{equation}
We assume $\omega_{ci}/\nu_i\ll1$ and neglect the $\vec{W}^{(1)}$, $\vec{W}^{(2)}$, $\vec{W}^{(3)}$, $\vec{W}^{(4)}$ terms. \citet{Mocanu2008} shows that the linearised viscosity tensor is given by
\begin{equation}
\begin{aligned}
    \vec{\pi}_0 &= \eta_0\qty(\vec{\hat{B}}_0 \otimes \vec{\hat{B}}_0 - \frac{1}{3}\vec{I})Q \\
    % &= \eta_0\qty(\vec{\hat{z}} \otimes \vec{\hat{z}} - \frac{1}{3}\vec{I})Q \\
    &=\frac{\eta_0}{3}\begin{pmatrix}
    -1 & 0  & 0 \\
    0  & -1 & 0 \\
    0  & 0  & 2  \\
    \end{pmatrix} Q,
\end{aligned} 
\end{equation}
where $\eta_0$ is given by Equation \eqref{eq:braginskii_eta_0} and 
\begin{gather}
    \vec{\hat{B}}_0 = \frac{\vec{B}_0}{B_0}, \\
    Q = 3\vec{\hat{B}}_0\vdot\grad(\vec{\hat{B}}_0\vdot\vec{u}) - \div{\vec{u}}
\end{gather}
We will assume our variables are of the form
\begin{equation}
    \label{eq:apdx_exp_form}
    f(x,t) = f_0\exp[i(k_x x + \omega t)],
\end{equation}
where $\omega\in \mathds{R}^+$ and $k_x\in \mathds{C}$. This simulates waves, propagating/decaying radially outwards/inwards from the solar surface at the equator. Enforcing $\pdv*{}{y}=\pdv*{}{z}=0$ causes
\[\pdv{}{t}\vec{u}\vdot\vec{\hat{y}}=\pdv{}{t}\vec{u}\vdot\vec{\hat{z}}=\pdv{}{t}\vec{\hat{b}}\vdot\vec{\hat{z}}=\pdv{}{t}\vec{\hat{b}}\vdot\vec{\hat{y}}=0\] 
and we let
\begin{gather}
    \vec{u} = u_x \vec{\hat{x}}, \\
    \vec{\hat{b}} = \hat{b}_z \vec{\hat{z}}.
\end{gather}
Hence,
\begin{gather}
    Q = -\pdv{u_x}{x}, \\
    \label{eq:ux_eqn1}
    \pdv{u_x}{t} = -v_A^2\pdv{\hat{b}_z}{x} + \nu_0\pdv[2]{u_x}{x}, \\
    \label{eq:bz_eqn1}
    \pdv{\hat{b}_z}{t} = -\pdv{u_x}{x}.
\end{gather}

We model the system as weakly viscous and assume that
\begin{equation}
\begin{aligned}
    \epsilon &= \frac{\nu_0 \omega}{v_A^2} \\
    &\approx 1.13\times10^{-1}\qty(\frac{\ln\Lambda}{20})^{-1}\qty(\frac{T}{3\times 10^6\si{.K}})^{5/2}\qty(\frac{\rho}{10^{-13}\si{.kg.m^{-3}}})^{-1} \\
    & \qquad\qquad\qquad\qty(\frac{\omega}{\pi\times10^{-2}\si{.s^{-1}}})\qty(\frac{v_A}{4\times10^5\si{.m.s^{-1}}})^{-2} \\
    &\ll 1.
\end{aligned}
\end{equation}

\section{Damping length}

In \citet{Withbroe1977,Parker1991} they suggest that the waves need to decay on a length scale of 1-2$R_\odot$ to play a significant role in coronal heating, where $R_\odot$ denotes the solar radius. This section aims to to derive a dispersion relation for the waves in our model to calculate a damping length. Taking the time derivative of \eqref{eq:ux_eqn1} to eliminate $\hat{b}_z$ gives
\begin{equation}
    \pdv[2]{u_x}{t} = v_A^2\pdv[2]{u_x}{x} + \nu_0\pdv{}{t}\pdv[2]{u_x}{x}.
\end{equation}
Dividing through by $v_A^2$ and assuming $u_x$ is of the form given by Equation \eqref{eq:apdx_exp_form} gives
\begin{equation}
    \frac{\omega^2}{v_A^2} = (1+i\epsilon)k_x^2.
\end{equation}
Using the binomial expansion gives
\begin{equation}
    k_x^2 = \frac{\omega^2}{v_A^2}\qty[1 -i\epsilon + O(\epsilon^2)].
\end{equation}
We assume $k_x$ is given by the negative root as the waves propagate away from the solar surface in the positive $x$-direction. Hence,
\begin{equation}
    k_x = -\frac{\omega}{v_A}\qty[1 -\frac{1}{2}i\epsilon + O(\epsilon^2)].
\end{equation}
The decay length-scale is given by
\begin{equation}
    \label{eq:apdx_decay_length}
    \begin{aligned}
    L_d &= \frac{2\pi}{\Im(k_x)} \\
    &\approx4\pi\frac{1}{\epsilon}\frac{v_A}{\omega} \\
    &= 4\pi\qty(\frac{1}{\nu_0})\frac{v_A^3}{\omega^2} \\
    &= 4\pi\qty(\frac{3}{2.21}\times10^{16}\ln\Lambda\, T^{-5/2}\rho)\frac{v_A^3}{\omega^2}\\
    &\approx 1.42\times10^9\qty(\frac{\ln \Lambda}{20})\qty(\frac{T}{3\times10^6\si{.K}})^{-5/2}\qty(\frac{\rho}{10^{-13}\si{.kg.m^{-3}}})\qty(\frac{v_A}{4\times10^5\si{.m.s^{-1}}})^3 \\
    &\qquad\qquad\qquad\qty(\frac{\omega}{\pi\times10^{-2}\si{.s^{-1}}})^{-2}\si{.m},
    \end{aligned}
\end{equation}
Equation \eqref{eq:apdx_decay_length} suggests that for typical coronal values, it is possible for fast waves to decay on a length-scale of about 2 solar radii, $R_\odot\approx 7\times10^8\si{.m}$. Our values for the density and Alfv\'en speed were chosen by using mean observed values and were taken from Table 1 in \citet{Morton2016}. Figure \ref{fig:power_spectrum_morton} shows that we chose a high frequency (where $\omega= 2\pi f$) and we discuss this further in the next section.

\section{Discussion}

We have shown that our model can dissipate fast waves on a length-scale of about 2 solar radii, and this suggests that this mechanism could play a significant role in coronal heating. However, we made many simplifications in our analysis, and it is unclear if relaxing these assumptions will increase or decrease the damping length. For example, we modelled the quantities as uniform and linear. This meant waves could not reflect and prevents reflection-driven turbulence \citep{Hollweg1986a,vanBallegooijen2011,Shoda2019}. Authors typically study Alfv\'en wave turbulence, but we think it could be interesting to investigate fast wave turbulence as these are compressible and therefore dissipate via viscosity more easily. We used a high frequency in \eqref{eq:apdx_decay_length} to account for the fact that waves may be able to cascade to higher frequencies due to turbulence. Including a more complex field can lead to several effects such as fast wave refraction towards null points \citep{McLaughlin2006,McLaughlin2011,McLaughlin2016}, the triggering of reconnection \citep{McLaughlin2009} and enhanced phase mixing \citep{Similon1989,Howson2020a}. We modelled the fast waves as propagating perpendicular to the field, in reality a component will propagate parallel and this could decrease the heating. Therefore, we would expect less heating at coronal holes where the field is approximately normal to the solar surface. In summary, the results provided here suggest further study could be fruitful. However, a more extensive analysis is needed to determine if the viscous dissipation of fast waves plays a significant role in coronal heating.